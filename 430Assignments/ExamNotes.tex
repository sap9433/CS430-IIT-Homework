\documentclass[0.5pt]{report}
\usepackage{amsmath,amsfonts,amssymb}
\usepackage[margin=0.4in]{geometry}
\usepackage{enumitem}
\usepackage{listings}
\newcommand{\floor}[1]{\lfloor #1 \rfloor}
\begin{document}
\noindent
\begin{itemize}
\setlength\itemsep{-.5em}
\item Height balance tree(AVL) Height $N_h = N_{h-1} + N_{h-2} + 1 $ , Is same as fibonacci So $ height = \log_\phi n $
\item In a Red black tree , 1) Each node is either red or black 2) Every leaf (Null nodes )is black. 3) The root is black 4) No red node has a red parent or a red child 4) Paths from the root to any leaf all pass through the same number of black nodes (Constant black depth).
\item RBT-Insertion (Always make new Insertion as Red)
~~\\If X's uncle is Red $\rightarrow $ Make X's parent and uncle black and X's grandparent red.  
~~\\If X's uncle is Black \& and X is the right chil $\rightarrow$rotate the edge between X \& its parent in opposite of X(yield Case 3)
~~\\ If X's uncle is Black and and X is the lef child $ \rightarrow$ rotate the edge between X's parent and X's Grand parent in Opposite of X. Color X's parent black and its old grandparent red.
\item Complexity Insertion - 1) Insert.O($\log n$) + Color.O(1) + Fix Violation.$(O(\log n) \times ( Recolor.O(1) + Rotation.O(1)) = O(2 \log n)$  
\item 
\begin{lstlisting} [basicstyle=\tiny]  
int LCSubStr(char *X, char *Y, int m, int n) // Longest common substring
{
    int LCSuff[m+1][n+1];
    int result = 0; 
    for (int i=0; i<=m; i++)
    {
        for (int j=0; j<=n; j++)
        {
            if (i == 0 || j == 0)
                LCSuff[i][j] = 0; //For simplicity . Doesn't mean anything
 
            else if (X[i-1] == Y[j-1])
            {
                LCSuff[i][j] = LCSuff[i-1][j-1] + 1;
                result = max(result, LCSuff[i][j]);
            }
            else LCSuff[i][j] = 0;
        }
    }
    return result;
}
\end{lstlisting}

\item
\begin{lstlisting} [basicstyle=\tiny]  
def lcs(X, Y, m, n):  //Longest common subsequence
 
    if m == 0 or n == 0:
       return 0;
    elif X[m-1] == Y[n-1]:
       return 1 + lcs(X, Y, m-1, n-1);
    else:
       return max(lcs(X, Y, m, n-1), lcs(X, Y, m-1, n));
\end{lstlisting}

\item
 \begin{lstlisting} [basicstyle=\tiny]
# Returns the maximum value that can be put in a knapsack of capacity W
def knapSack(W, wt, val, n): #DP 0-1 KnapsackProblem.
    K = [[0 for x in range(W+1)] for x in range(n+1)]
 
    # Build table K[][] in bottom up manner
    for i in range(n+1):
        for w in range(W+1):
            if i==0 or w==0:
                K[i][w] = 0
            elif wt[i-1] <= w:
                K[i][w] = max(val[i-1] + K[i-1][w-wt[i-1]],  K[i-1][w])
            else:
                K[i][w] = K[i-1][w]
 
    return K[n][W]
    
#val = [60, 100, 120] , wt = [10, 20, 30] , W = 50 ,n = len(val)
#print knapSack(W , wt , val , n)
 \end{lstlisting}
 
 \item
 
  \begin{lstlisting} [basicstyle=\tiny]
def minCoins(coins, m, V):  # Min Coin Change ,m is size of coins array 
    if (V == 0): # base case
        return 0
    res = sys.maxsize  # Initialize result
    
    for i in range(0, m): # Try every coin that has smaller value than V
        if (coins[i] <= V):
            sub_res = minCoins(coins, m, V-coins[i])
 
            # Check for INT_MAX to avoid overflow and see if  result can minimized
            if (sub_res != sys.maxsize and sub_res + 1 < res):
                res = sub_res + 1
    return res
 #print("Minimum coins required is",minCoins([9, 6, 5, 1], len(coins),11)) # 2
 \end{lstlisting}

 \end{itemize}
 \end{document}
