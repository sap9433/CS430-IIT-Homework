\documentclass[5pt]{article}
\usepackage{amsmath,amsfonts,amssymb}
\usepackage{listings}

\title{Solution to Homework Assignment 1 (CS 430)}
\author{Saptarshi Chatterjee \\
\texttt{CWID: A20413922}
}

\begin{document}
\maketitle

\section{Solution to Problem 2.3-3 on page 39}
\textbf{Q. Use mathematical induction to show that when n is an exact power of 2, the solution of the recurrence}

\begin{equation*}
  T(n) =\begin{cases}
    2, & \text{if $n = 2$},\\
    2T(n/2) + n, & \text{ $n = 2^k $, for $k > 1$}
  \end{cases}
\end{equation*}

\textbf{is $T(n) = n \log(n)$}

\setlength{\parskip}{1.2em}
\setlength{\parindent}{0em}

\textbf{Answer -}

Let $n = 2^k$  (As n is in power of 2 , given )
\begin{align*}
T(n) & = T(2^1)  & \text{(When $ k = 1 $ )}\\
& = T(2) \\
& = 2  & (\text{From given recurrence when $n = 2$})\\
& = 2 \log 2
\end{align*}
Now we need to prove given $T(2^k) = 2^k \log(2^k)$ holds true , then $T(2^{k+1}) = 2^{k+1} \log(2^{k+1})$ should hold true as well.

\begin{align*}
T(2^{k+1}) & = 2T(2^{k+1}/2) + 2^{k+1}  & (\text{From given recurrence when $k > 1$})\\
& = 2T(2^k) + 2^{k+1}\\
& = 2 \times 2^k \log(2^k) + 2^{k+1} & (\text{by inductive hypothesis})\\
& = 2^{k+1}( \log 2^k + 1 ) \\
& = 2^{k+1}( \log 2^k + \log 2) \\
& = 2^{k+1} \log 2^{k+1} & \text{\ldots hence proved}\\
\end{align*}

\section{Solution to Problem 2.3-4 on page 39}
\textbf{Q. We can express insertion sort as a recursive procedure as follows. In order to sort $A[1 \ldots n]$ , we recursively sort $A[1 \ldots n-1]$  and then insert $A[n]$  into the sorted array $A[1 \ldots n-1]$ . Write a recurrence for the running time of this recursive version of insertion sort.}

\setlength{\parskip}{1.2em}
\setlength{\parindent}{0em}

\textbf{Answer -}
An algo for the recursive procedure will be - 
\begin{lstlisting}
def recursive_insertion(result, n):
    if n > 0:
        recursive_insertion(result, n - 1)
        for i in range(0, n):
            if result[i] > result[n]:
                result_n = result[n]
                del result[n]
                result.insert(i, result_n)
                break
        return result
    else:
        return result[n]


data = [12, 15, -23, -4 , 6, 10, -35, 28]
print(recursive_insertion(data, len(data) - 1))
\end{lstlisting}
 \emph{From the above code  the recurrence relation will be -}
 
 \begin{equation*}
  T(n) =\begin{cases}
    1, & \text{if $n = 1$},\\
    T(n - 1) + n, & \text{ $n > 1$}
  \end{cases}
\end{equation*}

\emph{Solving the recurrence we get \cite{insertionalgo} -}
\begin{align*}
T(n) & = T(n-1) + n ~~~  (\text{From given recurrence when $n > 1$})\\
& = T(n-2) + (n-1) + n \\
& = \dots \\
& = n + (n-1) + (n-2) + \dots + 1 ~~~ (\text{given, when $n = 1 , T(n) = 1 $})\\
& = n(n+1) / 2  ~~~~ \text{( sum of n natural numbers)}\\
& = \Theta(n^2)
\end{align*}

 
 

\section{Solution to Problem 2-3(a) on page 41}
\textbf{Q. In terms of  $\Theta$ notation, what is the running time of  code fragment for Horner's rule?}

\setlength{\parskip}{1.2em}
\setlength{\parindent}{0em}

\textbf{Answer -}

Given the given the coefficients $a_0, a_1, \dots a_n $ (stored as an array) an and a value for $x$ , algo for the Horner's rule \cite{horneralgo} -
 
\begin{lstlisting}
def evaluate(x, a):
    result = 0
    for i in range(len(a)-1, -1, -1):
        result = a[i] + (x * result)
    return result
\end{lstlisting}

As the number of iterations is same as the value of n (length of the array storing values of 'a' ) , it has an asymptotically tight bound of $n$ . So complexity of the code fragment would be $\Theta(n)$ 

\section{Solution to Problem 3-3(a), fourth row only, on pages $61-62$. Justify your answers!}
\textbf{Q.Rank these functions $2^{\lg n} , (\lg n)^{\lg n}, e^n , 4^{\lg n}, (n+1)!, \sqrt{\lg n} $ by order of growth; that is, find an arrangement $g_1, g_2, \dots g_3$ of the functions satisfying  $ g_1 = \Omega(g_2)$ ,  $g_2 = \Omega(g_3) $ \dots}

\setlength{\parskip}{1.2em}
\setlength{\parindent}{0em}

\textbf{Answer -}


 $ g(n) = (n+1)! =  \Omega(e^n) $  ~~ \dots \\
 ( Analysis - \text{$ e^n = e \times e \times e ... n$ times} \\
 \text{ $(n+1)! = (n+1) \times n \times (n-1) \dots \times 1 $ . As e is roughly $2.71828$ , most of these terms }
 \text{ will be greater than e , when $ n > 4 $ . $(n+1)!$ is definitely greater than $e^n$ for large n  }) \\
 
 $g(n) = e^n = \Omega( (\lg n)^{\lg n} ) $ ~~ \dots \\
  (\text{ Analysis - If we take $\ln$ base e for both the sides we get R.H.S. as $\log n \ln_e {\log n}$ which is a function of } \\
  \text{logarithmic growth , where an L.H.S becomes n . Which is linear growth function.  }\\
  \text{Linear growth is faster than logarithmic growth})\\
 $ g(n) = (\lg n)^{\lg n} = \Omega( 4^{\lg n} )  $ \\
  (\text{ Analysis - R.H.S increases only in the power of 4 , which is a constant . }\\
  \\text{So there is definitely an n for which $\log n > 4$ })\\
 $ g(n) = 4^{\lg n} = \Omega( 2^{\lg n} ) $ \\ 
 (\text{ Same as above . Both grows in same power, but LHS grows in the order of 4  }\\
 \text{and RHS is in the order of 2} \\
$ g(n) = 2^{\lg n} = \Omega(\sqrt{\lg n})  $  \dots (\text{LHS grows exponentially so faster , RHS have $\sqrt{}$  of logarithmic })


\section{Problem 4-3(a) on page 108; solve this problem two ways: first with the master theorem on page 94, and then using secondary recurrences (page 13 in the January 10 notes)}
\textbf{Q. Find asymptotic tight bound for $T(n) = 4T(n/3) + n\log n $}

\setlength{\parskip}{1.2em}
\setlength{\parindent}{0em}

\textbf{Answer -}

\textbf{ Using master theorem } 

If we write the given recurrence as $ af(\frac n b) + f(n) \ $ , then we get $ f(n) = n \log n,  a = 4 , b = 3 $ . 
Now - 
\begin{align*}
\frac {a . f(\frac n b)} {f(n)}  & = \frac { \frac {4n} 3 \log {\frac n 3}}  {n \log n} \\
& =\frac 4 3   \frac { (\log {\frac n 3})}  {\log n} \\
& >  1 ~~~ ( \dots \text{for sufficiently large n} )
\end{align*}
Which satisfies 2nd rule of master's theorem as per lecture notes.
\begin{align*}
T(n) & = \Theta(n^{log_b a}) \\
& = \Theta(n^{log_3 4}) \\ \\
\end{align*}


\textbf{ Using secondary recurrence }\\

Given recurrence is  $T(n) = 4T(n/3) + n\log n $

Lets the base case be $ T(1) = 1 $ 

Lets take a secondary sequence $n_i$ such that 
$T (n_i) = 4T (n_{i-1}) + n_i \log n_i$

So $n_i$ is the argument of T() when we are i recursion steps away from the base case $n_0 = 1$. The original
 recurrence gives us the following secondary recurrence for $n_i$ : 
 
 $n_{i-1} = \frac {n_i}  3 $ , \dots implies $ n_i = 3n_{i-1} $
 
 The annihilator for this recurrence is $(E - 3) $, so the generic solution is $n_i = \alpha 3^i $. Plugging in
the base cases $n_0 = 1$ and $n_1 = 3$, we get the exact solution $n_i = 3^i $
 
 Notice that our original function $ T (n) $ is only well-defined if $ n = n_i $ for some integer $ i \geqslant 0 $ . Now to solve the original recurrence, we do a range transformation. If we set $t_i = T(n_i)$, we have the recurrence $t_i = 4 t_{i-1} + 3^i \log {3^i} $. The annihilator of the recurrence is $(E - 4)(E - 3)$ , so the generic solution is $ \alpha 4^i + \beta 3^i $. Plugging in the base cases$t_0 =1,t_1 = 7$  we get $\alpha = 4 , \beta = -3 $ the exact solution 
 $$
  t_i = 4^{i+1} - 3^{i+1} 
 $$

Finally, we need to substitute to get a solution for the original recurrence in terms of n, by inverting the solution of the secondary recurrence. If $n_i = 3^i $ , then (after a little algebra) we have $ i = log_3 n$

Substituting this into the expression for ti, we get our exact, closed-form solution.

\begin{align*}
T(n)  & = 4^{i+1} - 3^{i+1}  \\
& = 4 \times 4^{log_3 n} - 3 \times 3^{log_3 n} \\
& = 4 \times n^{log_3 4} - 3 \times n^{log_3 3} \\
& = \Theta(n^{log_3 4})
\end{align*}

\clearpage

\begin{thebibliography}{9}
\bibitem{horneralgo} 
Horner rule implimentation
\\\texttt{$https://introcs.cs.princeton.edu/python/21function/horner.py.html$}
 
\bibitem{insertionalgo} 
Insertion sort analysis
\\\texttt{$http://crypto.stanford.edu/~dabo/courses/cs161_spring01/cs161-02.pdf$}

\bibitem{treedraw} 
Tree Template
\\\texttt{$http://www.texample.net/tikz/examples/merge-sort-recursion-tree/$}

\end{thebibliography}



\end{document}


