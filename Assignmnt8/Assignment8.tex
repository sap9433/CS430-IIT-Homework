\documentclass[5pt]{article}
\usepackage{amsmath,amsfonts,amssymb}
\usepackage{listings}
\usepackage{graphicx}
\graphicspath{{/Users/diesel/Desktop/}}
\newcommand{\floor}[1]{\lfloor #1 \rfloor}

\title{Solution to Homework Assignment 8(CS 430)}
\author{Saptarshi Chatterjee \\
\texttt{CWID: A20413922}
}

\begin{document}
\maketitle


\section{Question 1. Problem 34.4-3 on page 1085}

\setlength{\parskip}{1.2em}
\setlength{\parindent}{0em}

\textbf{Answer -}

Lets say 3-CNF-SAT formula have x variables , then, the truth table corresponding
to this formula would have $2^x$ combinations.  The reduction step needs to
consider every possible values to all the variables. This then means that
the reduction process will need to iterate over all the rows of the Truth table , so complexity 
will be O($2^x$) . Hence, we can conclude that this strategy does not yield a polynomial-time reduction.


\section{Question 2.  Problem 34.4-7 on page 1086}

\setlength{\parskip}{1.2em}
\setlength{\parindent}{0em}

\textbf{Answer -}

\begin{itemize}
\item We will create a Graph where vertexes are variable v and $\neg {v}$
\item  We will draw an Edge ($\alpha,\beta$) iff there exists a clause in the 2-CNF-SAT as ($\neg\alpha\lor\beta$)
\item So, If there's an edge ($\alpha,\beta$) then there must be an edge ($\neg\alpha,\neg\beta$)
\item The 2-CNF formula  will be unsatisfiable if and only if there exists a variable v, such that there is a path from  (v and $\neg {v}$) and ($\neg {v}$ and v)

~~\\Proof:  Suppose there are path(s) (v and $\neg {v}$) and ($\neg {v}$ and v) and there exists a satisfying assignment for the 2-CNF-SAT

Case1: Let v =TRUE.
Let there be a path in the graph  be  $v  ....  \rightarrow \alpha  \rightarrow \beta \rightarrow .... \rightarrow \neg {v} $. The way we constructed the graph, there is an edge between (M,N)  iff there is a clause ($\neg M \lor N$). 
An edge from M to N represents that if M is TRUE, then N must be TRUE (for the clause to be TRUE). Now since v is true, all literals in path from ( v to $\alpha$ )must be TRUE. And all literals in the path from ($ \beta
to \neg {v}$ ) must be FALSE. This results in an edge between ($\alpha$ ,$\beta$) where $\alpha$ = TRUE and $\beta$ = FALSE. So   ($\neg\alpha \lor \beta$) becomes FALSE, contradicting our assumption
that there exists a satisfying assignment.
Case2: Let v = FALSE. similar as above

\item So if there is no such edge , here is we can assign boolean value -
\\a) pick an unassigned vertex and assign it T
\\b) Assign T to all reachable vertices
\\c) Assign F to their negations
\\d) Repeat until all vertices are assigned

\item So following algo can determine if $2-CNF-SAT \in P$
\\a) For each variable v find if there is a path from v to $\neg v$ and vice-versa.
\\b)  Reject if any of these tests succeeded.
\\c)  Accept otherwise
\\ Complexity of this Algo is O(n) so $2-CNF-SAT \in P$
\end{itemize}



\section{Question 3.a Prove that in any legal 3-coloring of the crossover gadget, the opposite corners are forced to have
the same color 3.b Prove that any assignment of colors to the corners such that opposite corners have the same color
extends to a legal 3-coloring of the entire crossover gadget .  3.c Use the following idea to prove that 3-colorability of planar graphs is NP-hard: Replace each point
at which another edge crosses edge (u, v) with a copy of the crossover gadget G.  }

\setlength{\parskip}{1.2em}
\setlength{\parindent}{0em}

\includegraphics[scale=.3]{3rd}
\includegraphics[scale=.3]{2nd}







\begin{thebibliography}{9}
\bibitem{solution}
CLSR Solution
\\\\texttt{http://sites.math.rutgers.edu/~ajl213/CLRS/Ch34.pdf}

\bibitem{solution}
2-SAT Solution
\\\\texttt{http://www.dei.unipd.it/~geppo/AA/DOCS/2SAT.pdf}

\bibitem{solution}
2-SAT IIT Solution
\\\\texttt{https://www.iitg.ac.in/deepkesh/CS301/assignment-2/2sat.pdf}

\bibitem{solution}
3 color
\\\\texttt{https://courses.cs.washington.edu/courses/cse431/14sp/scribes/lec15.pdf}

\end{thebibliography}

\end{document}


